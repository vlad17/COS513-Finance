%%%%%%%% ICML 2015 EXAMPLE LATEX SUBMISSION FILE %%%%%%%%%%%%%%%%%
%%%%%%%%%%%%%%%%%%%%%%%%%%%%%%%%%%%%%%%%%%%%%%%%%%%%%%%%%%%%%%%%%%

\documentclass{article}

\usepackage{times}
\usepackage{graphicx}
\usepackage{subfigure} 

% For citations
\usepackage{natbib}

% For algorithms
%\usepackage{algorithm}
%\usepackage{algorithmic}
% Packages hyperref and algorithmic misbehave sometimes.  We can fix
% this with the following command.
%\newcommand{\theHalgorithm}{\arabic{algorithm}}

\usepackage{hyperref}
\usepackage{amsmath}
\usepackage{amssymb}

\def \N {\mathbb{N}}
\def \Nbr {\mathcal{N}}
\def \Q {\mathbb{Q}}
\def \F {\mathbb{F}}
\def \then {\implies &}
\def \oif {\Longleftrightarrow &\,}
\def \given {\text{Given }&}
\def \assume {\text{Assume }&}
\def \thfr {\therefore &\enskip}
\def \bij {\leftrightarrow}
\def \inj {\rightarrowtail}
\def \sur {\twoheadedrightarrow}
\def \R {\mathbb{R}}
\def \C {\mathbb{C}}
\def \D {\mathbb{D}}
\def \iff {\Longleftrightarrow}
\def \kron {\boldsymbol\delta}
\def \id {\text{id}}

\def\Tx{\textbf{x}}
\def\Ty{\textbf{y}}
\def\quotient{\mathclose{}/\mathopen{}}
\def\Tf{\textbf{f}}
\def\Th{\textbf{h}}
\def\Tg{\textbf{g}}
\def\sumn{\sum_{n=0}^\infty}
\def\limn{\lim_{n\rightarrow\infty}}
\def\prodn{\prod_{n=0}^\infty}
\DeclareMathOperator\adj{adj}

\newcommand{\stc}[1]{\widetilde{#1}}   
\newcommand{\pa}[1]{ \left({#1}\right) }
\newcommand{\set}[2]{ \left\{ #1 \,\middle|\, #2 \right\} }
\newcommand{\shift}[1]{&\quad & \text{#1}\\}
\newcommand{\lem}[1]{\text{\textbf{L.\ref{#1}}}}
\newcommand{\card}[1]{\left\vert{#1}\right\vert}
\newcommand{\Ps}[1]{\mathcal{P}\left({ #1 }\right)}
\newcommand{\colv}[1]{\begin{pmatrix} #1 \end{pmatrix}}
\newcommand{\mat}[1]{\begin{pmatrix} #1 \end{pmatrix}}
\newcommand{\detmat}[1]{\begin{vmatrix} #1 \end{vmatrix}}
\newcommand{\spanb}[1]{\text{span}\{ #1 \}}
\newcommand{\abs}[1]{\left|#1\right|}
\newcommand{\Inner}[1]{\langle #1 \rangle}
\newcommand{\Innercpy}[1]{\langle #1, #1 \rangle}
\newcommand{\conj}[1]{{\overline{#1}}}

\DeclareMathOperator{\Tr}{tr}
\DeclareMathOperator{\Dim}{dim}
\DeclareMathOperator{\Rank}{rank}
\DeclareMathOperator{\Ker}{ker}
\DeclareMathOperator{\Diam}{diam}
\DeclareMathOperator{\Diag}{diag}
\DeclareMathOperator{\Int}{int}
\DeclareMathOperator{\Clo}{clo}
\DeclareMathOperator{\sgn}{sgn}
\DeclareMathOperator{\MyRe}{Re}
\DeclareMathOperator{\MyIm}{Im}
\DeclareMathOperator{\res}{res}

% Employ the following version of the ``usepackage'' statement for
% submitting the draft version of the paper for review.  This will set
% the note in the first column to ``Under review.  Do not distribute.''
%\usepackage{icml2015} 

% Employ this version of the ``usepackage'' statement after the paper has
% been accepted, when creating the final version.  This will set the
% note in the first column to ``Proceedings of the...''
\usepackage{icml2015}


% The \icmltitle you define below is probably too long as a header.
% Therefore, a short form for the running title is supplied here:
\icmltitlerunning{Commodities Forecasting with GDELT}

\begin{document}
\twocolumn[
\icmltitle{Commodities Forecasting from Non-Financial World News from GDELT}

% It is OKAY to include author information, even for blind
% submissions: the style file will automatically remove it for you
% unless you've provided the [accepted] option to the icml2015
% package.
\icmlauthor{Vladimir Feinberg}{vyf@princeton.edu}
\icmladdress{Princeton University}
\icmlauthor{Ghassen Jerfel}{gjerfel@princeton.edu}
\icmladdress{Princeton University}
\icmlauthor{Daway Chou-Ren}{dchouren@princeton.edu}
\icmladdress{Princeton University}
\icmlauthor{Zi Xiang Pan}{zpan@princeton.edu}
\icmladdress{Princeton University}
\icmlauthor{Tom Wu}{tongbinw@princeton.edu}
\icmladdress{Princeton University}

% You may provide any keywords that you 
% find helpful for describing your paper; these are used to populate 
% the "keywords" metadata in the PDF but will not be shown in the document
\icmlkeywords{GDELT, commodities, finance, forecasting, clustering, machine learning, ICML}

\vskip 0.3in
]

\begin{abstract} 

We propose a new method for predicting commodity price data through the use of aggregated daily non-macroeconomic news. Our method creates a static clustering to extract topics and weights each day's news events with an extracted importance metric. We use GDELT for raw news attributes and sentiment analysis. With a $K$-means model for clustering and logistic regression and a vector autoregressive model as a prediction for the probability that a commodity price will  increase the next day, and using an $80-10-10$ split on six years of news data from 2006-04-01 to 2015-02-08, we achieved an error rate of ??? with a ??? cutoff for binary classification.

\end{abstract} 

%TODO review below
% Submissions must be in PDF.
% The maximum paper length is \textbf{8 pages excluding references, and 10 pages including references} (pages 9 and 10 must contain only references).
% Do \textbf{not include author information or acknowledgments} in your initial submission. 
% Your paper should be in \textbf{10 point Times font}.
% Make sure your PDF file only uses Type-1 fonts.
% Place figure captions {\em under} the figure (and omit titles from inside the graphic file itself).  Place table captions {\em over} the table.
% References must include page numbers whenever possible and be as complete as possible.  Place multiple citations in chronological order.  
% Do not alter the style template; in particular, do not compress the paper format by reducing the vertical spaces.

\section{Introduction}
 
Financial data prediction is appealing both for its challenging nature and for its practical applications. Because market dynamics are complex and random, accurate price prediction is difficult. There are two types of market prediction techniques, fundamental analysis, which relies on an asset's data for forecasting, and technical analysis, which relies on historical trends to exploit market timing.\cite{schumaker2009textual} This paper examines the use of news data to augment technical analysis for prediction on commodity prices.

We focus on non-economic news data. The influence of economic news has been examined in the past (TODO: find source, see old blog post). In particular, we will analyze whether a summary of a day's news topics is predictive of commodity prices the next day, rather than a particular event. To avoid losing information about significance of particular events, we will rely on the sentiment analysis already applied to our dataset to evaluate the weight of each news item.

\subsection{Commodity Prediction}
Various studies have been done analysing the effect of news on stock prices\cite{mcqueen1993stock} and foreign exchange rates\cite{kamruzzaman2003svm}. Relatively little work has looked at applying news data to the prediction of the similar commodities market. Nevertheless, because commodities, by definition, must be extracted or produced by countries, it is likely that underlying factors of their production rates will be captured by local news, especially reporting on crisis events. Commodity prices are also sensitive to current conditions because the supply and demands that drive them are inelastic.\cite{chen2008can} This leads us to believe that real-time news information has predictive power for commodity prices. 

\subsection{GDELT Dataset}
We draw news from the Global Database of Events, Language, and Tone (GDELT), which aggregates news from broadcast, print, and web sources across 100 languages and parses out features to describe each item. Phua et al demonstrate the potential for using GDELT data to predict financial markets by predicting Singapore stock market prices. They use GDELT  to examine the impacts of the June 2013 Southeast Asian heat wave and the December 2013 Indian riots on Singapore.\cite{phua2014visual} As mentioned, since commodity prices are sensitive to current conditions, GDELT's day by day news aggregation is useful for predicting the next day's pricing data.

TODO: talk about importance vs. topic columns at a high level here (give a few ex of columns for each category)

 
\section{Motivating Analysis}


\subsection{GDELT Dataset}

\subsubsection{Sparsity in GDELT}

The space of news events spanned by all columns in GDELT is much larger than the subspace we expect news to lie on. There are likely to be at least two modes of low-dimensional interactions in the data: (1) that actors only interact within small cliques and (2) that each actor is involved in a small set of events. We conducted an initial analysis on a random sample of days before August 2015 to avoid making conclusions that overfit the test data.

\begin{figure}[ht]
\vskip 0.2in
\begin{center}
\centerline{\includegraphics[width=\columnwidth]{images/actors-per-actor.png}}
\caption{Box-and-whisker plot of number of distinct actors each CAMEO actor interacts with (each event may have up to 2 involved actors, where the first inflicts the action), performed on the day-stratified random sample of the events. The medium number of co-actors for both Actor1 and Actor2 categories is 3, with a Q3 of 9 and 10 and maximum of 988 and 990, respectively. The sample contained about 124K events total. The outliers with many interactions are generic or common names, such as \texttt{PRESIDENT} or \texttt{UNITED STATES}.
}
\end{center}
\vskip -0.2in
\label{fig:actors-per-actor}
\end{figure} 

As Figure \ref{fig:actors-per-actor} demonstrates, actor count is a heavily skewed distribution. This gives us confidence that actors are indeed in small cliques for the sampled days. We conduct a similar inspection for the number of unique CAMEO coded events per actor in Figure \ref{fig:events-per-actor}:

\begin{figure}[ht]
\vskip 0.2in
\begin{center}
\centerline{\includegraphics[width=\columnwidth]{images/events-per-actor}}
\caption{Number of events that occur for individual actors. The median actor encounters 3 events, and the 75\% most active ones still see less than 15. This diagram only shows the 95\% least active actors.}
\end{center}
\vskip -0.2in
\label{fig:events-per-actor}
\end{figure} 

Because of the sparsity that is present, we wish to reduce the dimension of our data for three reasons: 1) to more accurately represent it, 2) so we can generate models in a continuous space of reduced-dimension tuples of real values (instead of having some categorical values), and 3) so our data pipeline only has to handle a reduced data size.

We found that classical dimensionality reduction was not tractable to apply to a dataset of this size - the highly categorical nature of the dataset results in a large dimensional expansion when preparing numeric inputs to the algorithms. Because of this, and because of our dataset's observed sparsity, we turned to a clustering based approach. 

%\subsection{Bloomberg Commodity Prices}
%up and down spikes
%normally distributed about 0

\section{Related Work}

On the whole, predicting any financial market has proven to be difficult. Simon's work encapsulates many of the inherent problems. Price predictions might accurately predict directionality changes or general trends, but if these predictions consistently lag behind actual prices, they will not be useful despite a low RMSE. Model evaluation therefore must either demonstrate ability to generate profit through a trading scheme or demonstrate an ability to correctly predict directionality for desired trading period times and not merely a low RMSE. Simon identifies a training period of 3 years and a test period of 6 months as optimal for reducing predicted errors \cite{forex_neuralnets}.

\noindent In an examination of the potential for technical approaches to predict pricing movements using a Naive Bayes classifier, Gidofalvi found that a 20 minute period before and after the release of financial news allowed for a weak prediction of price movements \cite{gidofalvi2001using}. It was found by McQueen and Roley that fundamental macroeconomic news has little impact on stock prices but that other news types have effects dependent on responses of expected flows relative to equity discount rates \cite{mcqueen1993stock}.\\

\noindent Using textual analysis and SVMs for prediction, Schumaker and Chen were able to demonstrate 57\% directional accuracy when using breaking financial news articles to predict S\&P500 stock movements within 20 minute periods \cite{schumaker2009textual}. Bollen et al also achieved an accuracy of 86.7\% when incorporating semantic mood data from Twitter and using self-organizing fuzzy neural nets \cite{bollen2011twitter}.

\noindent A new area for price prediction is outlined by Phua et al, who validate the mostly non-financial GDELT as a valid news source for market prediction. This is in contrast to the financial data sets used by Schumaker and Chen, which is comprised of USecurity and Exchange Commission reports, stock related information such as from The Motley Fool, buy/sell/hold recommendations, and other similar news. Phua et al's topic analysis shows that GDELT consistently identifies impactful events on stock markets, despite its lack of financial focus. Term extraction from news sources is shown to be relevant using concept link exploration. However, Phua et al do find that not all significant events can be distinguished by GDELT. For example, the first riots in Singapore in 40 years did not appear significantly different than other news clusters. They were also unable to verify the quality of semantic score assignments. While they use decision trees to determine the factors potentially useful in price forecasting, they do not attempt to actually forecast prices in contrast to this paper.

\section{Methods and Evaluation}


Recent GDELT updates provide feeds with 15-minute resolution \cite{GDELT2}, but historical feeds offered only daily resolution. We decided for simplicity to analyze daily changes in price, taking a full day's worth of news data into account. This was most in line with testing our hypothesis, that an aggregation of topic clusters in a day's news is predictive of commodity pricing.

Our data pipeline was designed to provide day-wise parallelism over our data set in order to enable fast testing of new feature extraction methods. The raw data, as well as intermediate data, is stored in TSV ASCII format.

\subsection{Model}

Every news event in GDELT is stored as a row in a file for that day's report. Every set of rows may undergo a series of transformations. First, in the \textbf{preprocessing} stage, we extract relevant topic- and importance- related columns. For topics, a mix of numeric and categorical fields are extracted. Importance columns are saved for use in a further step. Next, in the \textbf{expansion} stage, we project each row of topic features to a purely real space by using one-hot encoding for categorical values. For each category, one-hot encoding produces $n$ dummy boolean variables, where $n$ is the number of unique categories. Finally, in the \textbf{summary} stage, $K$-means clustering is performed on this purely numeric representation of the data. Euclidean distance is used as the distance metric for $K$-means. Many clustering models are created with a range of $K$ from 10 to 5000. We use $K$-means for tractibility reasons.

The day summaries are then conglomerated in a time series $\{\textbf{d}_i\}$. For a day-indexed time series of a commodity's price, $\{p_i\}$, we have the corresponding sequence of binary labels indicating whether price has increased the next day, $y_i=1_{p_{i+1}>p_i}$.

Thus, our model assumes that:
\begin{enumerate}
\item The static clustering of news topics is accurate for the future.
\item $\mathbb{P}(p_{i+1}>p_i)$ may be determined by a regression over summaries $\textbf{d}_i$. 
\end{enumerate}

Both of these are strong assumptions. The first may be weakened by adapting a dynamic clustering model, such as an infinite Gaussian Mixture Model or a Hierarchical Dirichlet Process. The second requires $\textbf{d}_i$ to be both contain a sufficient amount of information to predict price behavior, which is a matter of appropriate feature extraction, and the assumption that we do not lose too much predictive power in summarizing a day by aggregating over the clusters to which its news events belonged.

\subsection{Execution}

Our current implementation uses $K$-means for clustering and logistic regression for classification. We trained, validated, and sampled from the days between 2006-01-01 and 2015-07-31. Recent days were used for testing.

For tractability reasons, we use a random sample of one million events to create the clustering models which are used in the main pipeline. This sample goes through the same preprocessing and expansion pipeline as the day summaries which are used in the regression except its output is used to create a clustering model. The sample is drawn uniformly from all events without any recency bias.

The raw data starts with $X=58$ columns. Each day has around $N=100,000$ rows. The total size of the set is about 60GB. %TODO precise mean and stddev.

\begin{figure}[ht]
\vskip 0.2in
\begin{center}
\centerline{\includegraphics[scale=0.15]{images/preprocess_vertical.png}}
\caption{Preprocessing stage}
\end{center}
\vskip -0.2in
\label{fig:preprocess}
\end{figure}

The preprocessing step does standard data cleaning such as removing malformed rows but more importantly removes all but 19 of the original columns, and also groups the remaining into $T=12$ topic-related columns and $I=9$ importance columns. The removed columns have either redundant data or data deemed insufficiently relevant to commodity prediction. The topic columns are in a compressed format, with categorical variables represented as integers. String columns are sanitized and have stop words removed.

\begin{figure}[ht]
\vskip 0.2in
\begin{center}
\centerline{\includegraphics[scale=0.15]{images/expand_vertical.png}}
\caption{Expansion stage}
\end{center}
\vskip -0.2in
\label{fig:exapand}
\end{figure}

The expansion step one-hot encodes categorical features, such as the event CAMEO codes,  producing $n$ dummy boolean variables, where $n$ is the number of unique categories. The 19 feature columns are expanded to 938.

%TODO mention here that we do sample-mean and sample-std-dev-based scaling, once we actually do that

\begin{figure}[ht]
\vskip 0.2in
\begin{center}
\centerline{\includegraphics[scale=0.15]{images/cluster_and_transform_vertical.png}}
\caption{Summary stage}
\end{center}
\vskip -0.2in
\label{fig:summarization}
\end{figure}

As mentioned, the random sample is used to generate $K$-means models. These models are then used to cluster each day's data. We extract the $N$x$K$ clustered data and separate it into two matrices containing clustered topic data and clustered importance data and pad these matrices with $1$s, as in \ref{summarization}. We then multiply the tranposed topic and importance matrices together. Because of the padded $1$s, their multiplication conveniently extracts the sum of importance-weighted events belonging to certain topics. Note that this model is extensible to a mixture model, where each event's contribution to a topic's importance for that day is weighed by the probability of belonging to that topic's class. Furthermore, the aggregation accomplished by the multiplication creates a highly reduced and uniform description of every day which is a flattened vector of dimension dependent on the cluster count.

These daily summaries are then enriched with historical pricing data before being fed into the linear model. The pricing data added are the 5, 10, and 30 day rolling averages of the commodity price. %TODO hyperparam selection of rolling mean.


\subsubsection{Autoregressive Models: ARMA}
%TODO ghassen add a transitioning sentence
We started our exploratory analysis with binary classification. We used diverse classification techniques such as Linear Regression and SVM on the sign of the change of the returns from a day to the next. However, our accuracy results were sub-par. Therefore, we decided to employ Autoregressive models which combine the news data with the time series data in order to produce continuous predictions of next-day time series values. %%TODO: TOM
Autoregressive models are some of the most flexible and easy to use models for time series. The future value of a variable, in an AR model, is assumed to be a linear function of several past observations and random errors. Accordingly, AR models have proven to be especially useful for describing the dynamic behavior of economic and financial time series and for forecasting \cite{tsay, VAR}. Recent literature proved their superiority in financial modeling to most other techniques such as Neural Nets and SVM. %%TODO: G reference
\\
The basic p-lag vector autoregressive model (AR(p)) has the form: $$\bf{Y}_t=\bf{c}+\bf{\Pi}_1\bf{Y}_{t-1}+\bf{\Pi}_2\bf{Y}_{t-2}+\cdots+\bf{\Pi}_p\bf{Y}_{t-p}+\epsilon_t,$$ $$t=1\ldots T$$ with $\Pi_i$ being the coefficients and $\epsilon_t$ is an unobservable zero mean noise. We can also do subset-autoregression where we pick multiple specific lags instead of a full range of lags like the basic AR(p) would do as show in the previous equation.
One of the most common extensions to Autoregressive models is to include moving averages and integration terms that take into account the stationarity of the time series. This is what constitutes Auto-Regressive Integrated Moving Average (ARIMA) models. They are  the most general class of models for forecasting a time series which can be made to be “stationary” by differencing (if necessary), perhaps in conjunction with nonlinear transformations.
Accordingly, an ARIMA model is fully determined by 3 parameters, one for each of the (AR),(I) and (MA) components: %%TODO: formatting
* p = order of autocorrelation for (AR)
* d = order of integration (differencing) for (I)
* q = order of moving averages for (MA)
Determining these orders can be done either visually using the Autocorrelation Function plot (finds the autocorrelation coefficients mentioned earlier) and the Partial Autocorrelation Function plot (correlation conditioned on other lags' autocorrelation) [Figure \ref{fig:ACF}] or analytically using a goodness of fit measure (which order fits the data best).
ACF or PACF plots indicates which lags are most correlated within the time series. They can be used to gauge if there is any sort of autocorrelation within the time series. They can be also be used to determine the exact orders depending on how the values in the plots decay or simply cut off but we're not covering those methods as we focused on the more methodical analytical approach.
\begin{figure}[ht]
	\vskip 0.2in
	\begin{center}
		\centerline{\includegraphics[width=\columnwidth]{ACF.PNG}}
		\caption{Autocorrelation Function plot for gold. Lags exceeding the dotted blue line have a statistically significant correlation with the 0-lag.Accordingly, the lags at 1, 14, 20 and 21 are correlated with the 0-lag (current time series).}
	\end{center}
	\vskip -0.2in
	\label{fig:ACF}
\end{figure}
For the goodness of fit, we had the options to use the Akaike information criterion (AIC), the Bayesian information criterion (BIC) or several other criterions that are more or less similar. These criterions would allow us to comapare the goodness of fit of different models, indexed by their parameters. Accordingly, we would be able to decide the best order or parameters for our model based on the set that achieves the best fit. Later, we will discuss which subset of the data this fit was tested on.
%%TODO G: Add the equations for AIC/BIC
For this paper, We decided to use the AIC which aims to choose a model that minimizes the KL divergence between the true density and the density of the MLE of a fixed model. Accordingly, the best model is the one with the lowest AIC. However, AIC is prone to overfitting and chooses predictive models over parsimonious ones. Given the literature's emphasis on the lack of predictive power in the domain of financial prediction, we decided to favor the predictive capacity over sparseness, for the order estimation, at least. BIC, on the other hand, favors sparsity over the predictive capabilities.
However, before testing the goodness of fit of different combinations of p and q parameters, we can figure out the d parameter using an Augmented Dickey Fuller test to verify the non-stationarity hypothesis of our data. Notice that random variable is stationary if its statistical properties are constant over time. Accordingly, it would have no trends and all of the variations around its mean would be random noise of constant amplitude. \cite{tsay, VAR}
If our data is already stationary, then there is no need for differencing (determining by the d component). In our case, for example, the non-stationary hypothesis was rejected with a p-value of 0.01 in the ADF test. Therefore, the d parameter that best fits our data is 0.
\subsubsection{Possible Extensions and Optimizations}
One possible extension to ARIMA or any time series model, for that matter, is adding a Markov Switching Model that assumes the existence of latent regimes and builds an independent model for each regime because.\cite{MS} After testing, we couldn't find any significant regimes in our training data which might infer that our regimes are over longer periods of time than that spanned by our data.
A possible optimization, which we benefited from, is the use of simple linear models (or generalized linear models) to simulate the AR component of ARIMA in case the (MA) component was deemed superfluous. In this case, we can build a simple design matrix where each column is a lagged version of the response variable. This design matrix can be used for a standard Ordinary Least Squares regressions as well as various other models that are computationally faster than ARIMA (with respect to their R implementations, at least).
\subsubsection{News Data: Exogenous Factors}
Our news data that we extract from GDELT is considered an exogenous factor. In the time series literature, there have been several approaches to including such information into time series models. The most two prominent methods are impulse responses or simply considering the exogenous data as time series to be evaluated alongside the financial times series (the way you'd do in a regression) such as in ARIMAX models (ARIMA + Exogenous).
The impulse response method attempts to detect impulses, whose response or effect decays with time,  within the time series data in order to localize the date of origin. Consequently, one would try to match the dates of such impulses with our news data in order to detect which columns in our data are most correlated with the big changes in those dates.
ARIMAX is simpler and more efficient, in literature and from our experience, especially with our linear models simplification. We add the columns of the news as extra time series whose autocorrelation with the response variable is evaluated in training. Note however that we use a lagged version of the news since our hypothesis is that the news from day D would affect the market after day D's closing which will be reflected on the difference of the values between D+1 and D.
\subsubsection{Column Selection of News Data}
However, a large number of columns in the news data (800 in the case of 100 clusters of GDELT data) can easily cause overfitting since the large number of variables could fit arbitrary data points to the noisy news columns. Accordingly, our out-of-sample testing results would suffer.
Another problem with a large number of columns is that ARIMA (the original implementation) and Ordinary Least Squares regression require the number of features to be less than or equal to the number of observations. This can be limiting as we would like to test different numbers of trainign days. Therefore, we recurred to Principal Component Analysis (among other methods that we tried) to reduce the dimensions. PCA converts a set of observations of possibly correlated variables into a set of values of linearly uncorrelated variables called principal components.\cite{PCA} One can select a fixed number of principal components that express most of the variance in a certain data set. In our case, we selected a number of principal components that accounted for $99\%$ of the variance in our data and used those principal components of the news data for the training as exogenous factors, instead of the original news columns.
\subsubsection{L1-regularizatio: Lasso Regression}
We also tried L1-regularization with Lasso regression since it inherently handles overfitting by enforcing sparsity, as compared to pre-emptively regularizing based on the variance, as in the case of PCA. \cite{glmnet} Lasso's sparsity is based on the actual importance of each column to the response variable and allows us to select a small number of explanatory variables. Additionally, Lasso doesn't require the number of variables to be less than the observations which allows us to test a small number of observations without applying any dimension reduction beforehand.
%%TODO GHASSEN: Explain Lasso More
Additionally, Lasso is computationally efficient as compared to other sparse regressions.\cite{glmnet} Finally, we can select the degree of our sparsity by selecting a Lambda coefficient which helps us better gauge the change of the predictive power of our model.
Accordingly, we build Lasso model with the minimum Lambda as defined by cross-validation and then use that model to estimate our log-returns for each training window.
\subsubsection{Testing Measures}
For out-of-sample testing, we used two measures to evaluate the prdictive capacity of the model and if it suffers from bias caused by the training data. The first is Mean Absolute Error which describes the difference in absolute value between our predictions and actual values. The closer MAE is to 0, the better we fit the data quantitatively. However, for our final purpose of designing a trading strategy, it could be good enough to detect if the stock is going up or down. In this case, predicting the binary values is what we seek. Accordingly, we would convert our predictions to binary based on their sign (positive log returns implies an increase from last day) and use simple binary-classification measures such as the accuracy rate. We focused on the accuracy rate because we have almost equal amount of positive and negative data points that we are not biased by the training which is usually the reason for evaluating recall, precision and F-1 measures..
\subsubsection{Training the Time Series}
%%TODO G: DESCRIBE THE TIME SERIES: What are they? Origin? Scale? Median/kind of values? 
%%TODO G: DESCRIBE WHY WE'RE USING LOG RETURNS: 
We split our data into training and testing windows of varying sizes as can be seen in the results section. The windows guarantee the temporal order of the data points which is important for a time series analysis. We also resorted to a moving window technique, also called walk-forward optimization, which is a form of k-fold cross-validation for time series data. In this technique, we would slide the window on the training set of the time series data and re-estimate the coefficients at each iteration. Note that we're not re-estimating the order as we opted for estimating the order one time for the whole training dataset since we assumed that the order would be more or less constant and that re-computing it would be overly expensive and a big deterrent to any future online implementations of this model or system. 
Accordingly, moving-window training would have a higher predictive power than a simple fixed window technique since the model changes as the window moves and thus carries more weight as the predictions go further in time. With a fixed window, the further the predictions are from the window, the less relevant is the model from the fixed window since it's based on data that is too old. %%TODO GHASSEN: maybe reword this paragraph
Eventually, we ended up implementing a moving-window training strategy that looks ahead one step at a time and re-estimates the coefficients at each time but uses the same orders (number of lags and ma coefficients). We then collect the mean absolute error and the binary accuracy over all the iterations and average them to evaluate the model for a given hyper-parameter. In our case, the main hyper-parameters was the size of the training window.



\section{Results}


\subsection{Clustering}

We manually inspect our clusters for both high similary of articles within clusters and ow similarity of articles between clusters. Since we do not have a gold standard for news event labels, we cannot calculate values such as true positives and true negatives. We present some qualitative results.

\textbf{First ten articles of Cluster 0:}  
\begin{enumerate}
\item Retrial of 3 Al-Jazeera journalists (Egypt)
\item Protests for seizing of hospital accounts (DR Congo)
\item Waterloo homicide arrest (England)
\item Taliban vs Afghan army (Afghanistan)
\item \$20 million funding for Wilmington pharmacy (Delaware)
\item Tiger farms violate Endangered Species Law (China)
\item DeSoto corruption trial (Mississippi)
\item Six die at David Owuor's Nakuru crusade (Kenya)
\item Temp workers fight for wages (Chicago)
\item Grant of bail to Lakhvi, mastermind of Mumbai attack (Pakistan)
\end{enumerate}

As can be seen, purity for cluster 0 is fairly high, with seven of ten articles dealing with judicial decisions. In general, most clusters appear to have reasonable purity, but there is also high similarity between clusters. For example, articles similar to the first article in cluster 0 also appear in cluster 5. Cluster 5 is similar to cluster 0 in that they both deal with judicial decisions.\\

\noindent A search for the name of the mastermind of the Mumbai attacks, Zakiur Rehman Lakhvi, returned results in 69 of 1000 clusters, which indicates we have much more cross-cluster similarity than we'd like. But this is a more preferable problem to have than a lack of intra-cluster similarity. Our eyeball test can only evaluate cluster quality based on keywords such as actor names and locations and general topics. Its possible that Lakhvi can appear in 69 clusters because the other metrics articles are clustered on, such as tone or the importance features, which an eyeball test cannot detect, do in fact separate Lakhvi related articles into that many clusters.

\subsection{Regression}



\begin{figure}[h!]
	\caption{Accuracy for various test sizes}
	\centering
	\includegraphics[width=0.5\textwidth]{results/acc.jpg}
\end{figure}

\begin{figure}[h!]
	\caption{Mean aboslute error for various test sizes}
	\includegraphics[width=0.5\textwidth]{results/mae.jpg}
\end{figure}

\begin{figure}[h!]
	\caption{Accuracy for various sparsity}
	\includegraphics[width=0.5\textwidth]{results/acc2.jpg}
\end{figure}

\begin{figure}[h!]
	\caption{Mean abolsute error for various sparsity}
	\includegraphics[width=0.5\textwidth]{results/mae2.jpg}
\end{figure}

\begin{figure}[h!]
	\caption{Comparison of log returns for sparsity = 0.1, training size = 400}
	\includegraphics[width=0.5\textwidth]{results/prediction.jpg}
\end{figure}


TODO(Sean): test err, $R^2$, residuals plot, residuals normality test?
acc/mae
Accuracy across decreasing testing size(columns) and decreasing training size(row)
MAE(mean absolute error) same labelling
Fixed windw train last 500 test on 200
acc2/mae2
decreasing training size(columns) decreasing sparsity (row)
mae2(3,3) is lowest
moving windows train 200 test 

pred vs actual

best adj r2 is 0.66


% remember to add the disclaimer on the current results

%TODO remember to add runtime info later (not for first run)

%TODO what do our model's residuals look like (if using linear regression) - normal?

%TODO describe the distribution of the day-summaries - what does this high-dimensional feature vector distribution look like over the days? Is it very separable? Can we apply PCA, etc?

\section{Future Work}

Take advantage of time series progression somehow

HDP but how do we deal with a variable number of features? -> model would need to be fully bayesian to be able to learn new slopes for new clusters. At that point we're not even train/testing anymore we just start with nothing and run it through the days.

\subsection{Software and Data}

All code used is available in an open-source repository.\footnote{\hyperref[https://github.com/vlad17/COS513-Finance]{\texttt{https://github.com/vlad17/COS513-Finance}}} GDELT provides free access to its database as well.\footnote{ \hyperref[http://data.gdeltproject.org/events/index.html]{\texttt{http://data.gdeltproject.org/events/index.html}}}

Commodities pricing data is retrieved using the \texttt{R} package \texttt{quantmod}\footnote{\hyperref[http://www.quantmod.com/]{\texttt{http://www.quantmod.com/}}}, which pulls historical prices using Yahoo.

%TODO add a bunch more references everywhere to defend all our claims.

% Acknowledgements should only appear in the accepted version. 
% \section*{Acknowledgments} 

% In the unusual situation where you want a paper to appear in the
% references without citing it in the main text, use \nocite
% \nocite{langley00}

\bibliography{biblio}
\bibliographystyle{icml2015}


\end{document}