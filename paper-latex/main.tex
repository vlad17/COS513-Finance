
%%%%%%%% ICML 2015 EXAMPLE LATEX SUBMISSION FILE %%%%%%%%%%%%%%%%%
%%%%%%%%%%%%%%%%%%%%%%%%%%%%%%%%%%%%%%%%%%%%%%%%%%%%%%%%%%%%%%%%%%

\documentclass{article}

\usepackage{times}
\usepackage{graphicx}
\usepackage{subfigure} 

% For citations
\usepackage{natbib}
\usepackage[utf8]{inputenc}

% For algorithms
%\usepackage{algorithm}
%\usepackage{algorithmic}
% Packages hyperref and algorithmic misbehave sometimes.  We can fix
% this with the following command.
%\newcommand{\theHalgorithm}{\arabic{algorithm}}

\usepackage{hyperref}
\usepackage{amsmath}
\usepackage{amssymb}

\def \N {\mathbb{N}}
\def \Nbr {\mathcal{N}}
\def \Q {\mathbb{Q}}
\def \F {\mathbb{F}}
\def \then {\implies &}
\def \oif {\Longleftrightarrow &\,}
\def \given {\text{Given }&}
\def \assume {\text{Assume }&}
\def \thfr {\therefore &\enskip}
\def \bij {\leftrightarrow}
\def \inj {\rightarrowtail}
\def \sur {\twoheadedrightarrow}
\def \Z {\mathbb{Z}}
\def \R {\mathbb{R}}
\def \C {\mathbb{C}}
\def \D {\mathbb{D}}
\def \iff {\Longleftrightarrow}
\def \kron {\boldsymbol\delta}
\def \id {\text{id}}

\def\Tx{\textbf{x}}
\def\Ty{\textbf{y}}
\def\quotient{\mathclose{}/\mathopen{}}
\def\Tf{\textbf{f}}
\def\Th{\textbf{h}}
\def\Tg{\textbf{g}}
\def\sumn{\sum_{n=0}^\infty}
\def\limn{\lim_{n\rightarrow\infty}}
\def\prodn{\prod_{n=0}^\infty}
\DeclareMathOperator\adj{adj}

\newcommand{\stc}[1]{\widetilde{#1}}   
\newcommand{\pa}[1]{ \left({#1}\right) }
\newcommand{\set}[2]{ \left\{ #1 \,\middle|\, #2 \right\} }
\newcommand{\shift}[1]{&\quad & \text{#1}\\}
\newcommand{\lem}[1]{\text{\textbf{L.\ref{#1}}}}
\newcommand{\card}[1]{\left\vert{#1}\right\vert}
\newcommand{\Ps}[1]{\mathcal{P}\left({ #1 }\right)}
\newcommand{\colv}[1]{\begin{pmatrix} #1 \end{pmatrix}}
\newcommand{\mat}[1]{\begin{pmatrix} #1 \end{pmatrix}}
\newcommand{\detmat}[1]{\begin{vmatrix} #1 \end{vmatrix}}
\newcommand{\spanb}[1]{\text{span}\{ #1 \}}
\newcommand{\abs}[1]{\left|#1\right|}
\newcommand{\Inner}[1]{\langle #1 \rangle}
\newcommand{\Innercpy}[1]{\langle #1, #1 \rangle}
\newcommand{\conj}[1]{{\overline{#1}}}

\DeclareMathOperator{\Tr}{tr}
\DeclareMathOperator{\Dim}{dim}
\DeclareMathOperator{\Rank}{rank}
\DeclareMathOperator{\Ker}{ker}
\DeclareMathOperator{\Diam}{diam}
\DeclareMathOperator{\Diag}{diag}
\DeclareMathOperator{\Int}{int}
\DeclareMathOperator{\Clo}{clo}
\DeclareMathOperator{\sgn}{sgn}
\DeclareMathOperator{\MyRe}{Re}
\DeclareMathOperator{\MyIm}{Im}
\DeclareMathOperator{\res}{res}

% Employ the following version of the ``usepackage'' statement for
% submitting the draft version of the paper for review.  This will set
% the note in the first column to ``Under review.  Do not distribute.''
%\usepackage{icml2015} 

% Employ this version of the ``usepackage'' statement after the paper has
% been accepted, when creating the final version.  This will set the
% note in the first column to ``Proceedings of the...''
\usepackage[accepted]{icml2015}


% The \icmltitle you define below is probably too long as a header.
% Therefore, a short form for the running title is supplied here:
\icmltitlerunning{Commodities Forecasting with GDELT}

\begin{document} 

\twocolumn[
\icmltitle{Commodities Forecasting from Non-Financial World News from GDELT}

% It is OKAY to include author information, even for blind
% submissions: the style file will automatically remove it for you
% unless you've provided the [accepted] option to the icml2015
% package.
\icmlauthor{Vladimir Feinberg}{vyf@princeton.edu}
\icmladdress{Princeton University}
\icmlauthor{Daway Chou-Ren}{dchouren@princeton.edu}
\icmladdress{Princeton University}

% You may provide any keywords that you 
% find helpful for describing your paper; these are used to populate 
% the "keywords" metadata in the PDF but will not be shown in the document
\icmlkeywords{GDELT, commodities, finance, forecasting, clustering, machine learning, ICML}

\vskip 0.3in
]

\begin{abstract} 

%TODO fill in abstract.

\end{abstract} 

%TODO review below
% Submissions must be in PDF.
% The maximum paper length is \textbf{8 pages excluding references, and 10 pages including references} (pages 9 and 10 must contain only references).
% Do \textbf{not include author information or acknowledgments} in your initial submission. 
% Your paper should be in \textbf{10 point Times font}.
% Make sure your PDF file only uses Type-1 fonts.
% Place figure captions {\em under} the figure (and omit titles from inside the graphic file itself).  Place table captions {\em over} the table.
% References must include page numbers whenever possible and be as complete as possible.  Place multiple citations in chronological order.  
% Do not alter the style template; in particular, do not compress the paper format by reducing the vertical spaces.

\section{Introduction}
 
% Daway 
Financial data prediction is appealing both for its challenging nature and for its practical applications. Because market dynamics are complex and random, accurate price prediction is difficult. There are two types of market prediction techniques, fundamental analysis, which relies on an asset's data for forecasting, and technical analysis, which relies on historical trends to exploit market timing.\cite{schumaker2009textual} This paper examines the use of news data to augment technical analysis for prediction on commodity prices.

We focus on non-economic news data. The influence of economic news has been examined in the past (TODO: find source, see old blog post). In particular, we will analyze whether a summary of a day's news topics is predictive of commodity prices the next day, rather than a particular event. To avoid losing information about significance of particular events, we will rely on the sentiment analysis already applied to our dataset to evaluate the weight of each news item.

\subsection{Commodity Prediction}
Various studies have been done analysing the effect of news on stock prices\cite{mcqueen1993stock} and foreign exchange rates\cite{kamruzzaman2003svm}. Relatively little work has looked at applying news data to the prediction of the similar commodities market. Nevertheless, because commodities, by definition, must be extracted or produced by countries, it is likely that underlying factors of their production rates will be captured by local news, especially reporting on crisis events. Commodity prices are also sensitive to current conditions because the supply and demands that drive them are inelastic.\cite{chen2008can} This leads us to believe that real-time news information has predictive power for commodity prices. 

\subsection{GDELT Dataset}
We draw news from the Global Database of Events, Language, and Tone (GDELT), which aggregates news from broadcast, print, and web sources across 100 languages and parses out features to describe each item. Phua et al demonstrate the potential for using GDELT data to predict financial markets by predicting Singapore stock market prices. They use GDELT  to examine the impacts of the June 2013 Southeast Asian heat wave and the December 2013 Indian riots on Singapore.\cite{phua2014visual} As mentioned, since commodity prices are sensitive to current conditions, GDELT's day by day news aggregation is useful for predicting the next day's pricing data.

TODO: talk about importance vs. topic columns at a high level here (give a few ex of columns for each category)
 
\section{Motivating Analysis}

% Vlad

\subsection{GDELT Dataset}

\subsubsection{Sparsity in GDELT}

The space of news events spanned by all columns in GDELT is much larger than the subspace we expect news to lie on. There are likely to be at least two modes of low-dimensional interactions in the data: (1) that actors only interact within small cliques and (2) that each actor is involved in a small set of events. We conducted an initial analysis on a random sample of days before August 2015 to avoid making conclusions that overfit the test data.

\begin{figure}[ht]
\vskip 0.2in
\begin{center}
\centerline{\includegraphics[width=\columnwidth]{images/actors-per-actor.png}}
\caption{Box-and-whisker plot of number of distinct actors each CAMEO actor interacts with (each event may have up to 2 involved actors, where the first inflicts the action), performed on the day-stratified random sample of the events. The medium number of co-actors for both Actor1 and Actor2 categories is 3, with a Q3 of 9 and 10 and maximum of 988 and 990, respectively. The sample contained about 124K events total. The outliers with many interactions are generic or common names, such as \texttt{PRESIDENT} or \texttt{UNITED STATES}.
}
\end{center}
\vskip -0.2in
\label{fig:actors-per-actor}
\end{figure} 

As Figure \ref{fig:actors-per-actor} demonstrates, actor count is a heavily skewed distribution. This gives us confidence that actors are indeed in small cliques for the sampled days. We conduct a similar inspection for the number of unique CAMEO coded events per actor in Figure \ref{fig:events-per-actor}:

\begin{figure}[ht]
\vskip 0.2in
\begin{center}
\centerline{\includegraphics[width=\columnwidth]{images/events-per-actor}}
\caption{Number of events that occur for individual actors. The median actor encounters 3 events, and the 75\% most active ones still see less than 15. This diagram only shows the 95\% least active actors.}
\end{center}
\vskip -0.2in
\label{fig:events-per-actor}
\end{figure} 

Because of the sparsity that is present, we wish to reduce the dimension of our data for three reasons: 1) to more accurately represent it, 2) so we can generate models in a continuous space of reduced-dimension tuples of real values (instead of having some categorical values), and 3) so our data pipeline only has to handle a reduced data size.

We found that classical dimensionality reduction was not tractable to apply to a dataset of this size - the highly categorical nature of the dataset results in a large dimensional expansion when preparing numeric inputs to the algorithms. Because of this, and because of our dataset's observed sparsity, we turned to a clustering based approach. 

%\subsection{Bloomberg Commodity Prices}
%up and down spikes
%normally distributed about 0

\section{Related Work}

% Daway
On the whole, predicting any financial market has proven to be difficult. Simon's work encapsulates many of the inherent problems. Price predictions might accurately predict directionality changes or general trends, but if these predictions consistently lag behind actual prices, they will not be useful despite a low RMSE. Model evaluation therefore must either demonstrate ability to generate profit through a trading scheme or demonstrate an ability to correctly predict directionality for desired trading period times and not merely a low RMSE. Simon identifies a training period of 3 years and a test period of 6 months as optimal for reducing predicted errors \cite{forex_neuralnets}.

\noindent In an examination of the potential for technical approaches to predict pricing movements using a Naive Bayes classifier, Gidofalvi found that a 20 minute period before and after the release of financial news allowed for a weak prediction of price movements \cite{gidofalvi2001using}. It was found by McQueen and Roley that fundamental macroeconomic news has little impact on stock prices but that other news types have effects dependent on responses of expected flows relative to equity discount rates \cite{mcqueen1993stock}.\\

\noindent Using textual analysis and SVMs for prediction, Schumaker and Chen were able to demonstrate 57\% directional accuracy when using breaking financial news articles to predict S\&P500 stock movements within 20 minute periods \cite{schumaker2009textual}. Bollen et al also achieved an accuracy of 86.7\% when incorporating semantic mood data from Twitter and using self-organizing fuzzy neural nets \cite{bollen2011twitter}.

\noindent A new area for price prediction is outlined by Phua et al, who validate the mostly non-financial GDELT as a valid news source for market prediction. This is in contrast to the financial data sets used by Schumaker and Chen, which is comprised of USecurity and Exchange Commission reports, stock related information such as from The Motley Fool, buy/sell/hold recommendations, and other similar news. Phua et al's topic analysis shows that GDELT consistently identifies impactful events on stock markets, despite its lack of financial focus. Term extraction from news sources is shown to be relevant using concept link exploration. However, Phua et al do find that not all significant events can be distinguished by GDELT. For example, the first riots in Singapore in 40 years did not appear significantly different than other news clusters. They were also unable to verify the quality of semantic score assignments. While they use decision trees to determine the factors potentially useful in price forecasting, they do not attempt to actually forecast prices in contrast to this paper.

\section{Methods}

% Vlad (+ Daway for pictures)


\begin{figure}[ht]
\vskip 0.2in
\begin{center}
\centerline{\includegraphics[width=\columnwidth]{images/cluster_and_transform_vertical}}
\caption{Summarization stage}
\end{center}
\vskip -0.2in
\end{figure} 

\section{Evaluation}

% Vlad

\section{Results}

% Daway
% remember to add the disclaimer on the current results

%TODO remember to add runtime info later (not for first run)

\section{Future Work}

% Vlad

\subsection{Software and Data}

All code used is available at \hyperref[https://github.com/vlad17/COS513-Finance]{''https://github.com/vlad17/COS513-Finance''}. GDELT provides free access to its database as well: \hyperref[http://data.gdeltproject.org/events/index.html]{''http://data.gdeltproject.org/events/index.html''}.

Commodities pricing data is retrieved using the R package quantmod, which pulls historical prices using Yahoo: \hyperref[http://www.quantmod.com/]{''http://www.quantmod.com/''}

%TODO add a bunch more references everywhere to defend all our claims.

% Acknowledgements should only appear in the accepted version. 
% \section*{Acknowledgments} 

% In the unusual situation where you want a paper to appear in the
% references without citing it in the main text, use \nocite
% \nocite{langley00}

\bibliography{biblio}
\bibliographystyle{icml2015}

\end{document} 

% EXAMPLE FOOTNOTE
%\footnote{Example footnote should have full sentences.}

% EXAMPLE IMAGES
%\begin{figure}[ht]
%\vskip 0.2in
%\begin{center}
%\centerline{\includegraphics[width=\columnwidth]{icml_numpapers}}
%\caption{Historical locations and number of accepted papers for International
%  Machine Learning Conferences (ICML 1993 -- ICML 2008) and
%  International Workshops on Machine Learning (ML 1988 -- ML
%  1992). At the time this figure was produced, the number of
%  accepted papers for ICML 2008 was unknown and instead estimated.}
%\label{icml-historical}
%\end{center}
%\vskip -0.2in
%\end{figure} 

%Citations within the text should include the authors' last names and
%year. If the authors' names are included in the sentence, place only
%the year in parentheses, for example when referencing Arthur Samuel's
%pioneering work \yrcite{Samuel59}. Otherwise place the entire
%reference in parentheses with the authors and year separated by a
%comma \cite{Samuel59}. List multiple references separated by
%semicolons \cite{kearns89,Samuel59,mitchell80}. Use the `et~al.'
%construct only for citations with three or more authors or after
%listing all authors to a publication in an earlier reference \cite{MachineLearningI}.

% EXAMPLE TABLE
%\begin{table}[t]
%\caption{Classification accuracies for naive Bayes and flexible 
%Bayes on various data sets.}
%\label{sample-table}
%\vskip 0.15in
%\begin{center}
%\begin{small}
%\begin{sc}
%\begin{tabular}{lcccr}
%\hline
%\abovespace\belowspace
%Data set & Naive & Flexible & Better? \\
%\hline
%\abovespace
%Breast    & 95.9$\pm$ 0.2& 96.7$\pm$ 0.2& $\surd$ \\
%Cleveland & 83.3$\pm$ 0.6& 80.0$\pm$ 0.6& $\times$\\
%Glass2    & 61.9$\pm$ 1.4& 83.8$\pm$ 0.7& $\surd$ \\
%Credit    & 74.8$\pm$ 0.5& 78.3$\pm$ 0.6&         \\
%Horse     & 73.3$\pm$ 0.9& 69.7$\pm$ 1.0& $\times$\\
%Meta      & 67.1$\pm$ 0.6& 76.5$\pm$ 0.5& $\surd$ \\
%Pima      & 75.1$\pm$ 0.6& 73.9$\pm$ 0.5&         \\
%\belowspace
%Vehicle   & 44.9$\pm$ 0.6& 61.5$\pm$ 0.4& $\surd$ \\
%\hline
%\end{tabular}
%\end{sc}
%\end{small}
%\end{center}
%\vskip -0.1in
%\end{table}