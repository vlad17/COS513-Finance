Financial data prediction is appealing both for its challenging nature and for its practical applications. Because market dynamics are complex and random, accurate price prediction is difficult. There are two types of market prediction techniques, fundamental analysis, which relies on an asset's data for forecasting, and technical analysis, which relies on historical trends to exploit market timing.\cite{schumaker2009textual} This paper examines the use of news data to augment technical analysis for prediction on commodity prices.

We focus on non-economic news data. The influence of economic news has been examined in the past (TODO: find source, see old blog post). In particular, we will analyze whether a summary of a day's news topics is predictive of commodity prices the next day, rather than a particular event. To avoid losing information about significance of particular events, we will rely on the sentiment analysis already applied to our dataset to evaluate the weight of each news item.

\subsection{Commodity Prediction}
Various studies have been done analysing the effect of news on stock prices\cite{mcqueen1993stock} and foreign exchange rates\cite{kamruzzaman2003svm}. Relatively little work has looked at applying news data to the prediction of the similar commodities market. Nevertheless, because commodities, by definition, must be extracted or produced by countries, it is likely that underlying factors of their production rates will be captured by local news, especially reporting on crisis events. Commodity prices are also sensitive to current conditions because the supply and demands that drive them are inelastic.\cite{chen2008can} This leads us to believe that real-time news information has predictive power for commodity prices. 

\subsection{GDELT Dataset}
We draw news from the Global Database of Events, Language, and Tone (GDELT), which aggregates news from broadcast, print, and web sources across 100 languages and parses out features to describe each item. Phua et al demonstrate the potential for using GDELT data to predict financial markets by predicting Singapore stock market prices. They use GDELT  to examine the impacts of the June 2013 Southeast Asian heat wave and the December 2013 Indian riots on Singapore.\cite{phua2014visual} As mentioned, since commodity prices are sensitive to current conditions, GDELT's day by day news aggregation is useful for predicting the next day's pricing data.

TODO: talk about importance vs. topic columns at a high level here (give a few ex of columns for each category)