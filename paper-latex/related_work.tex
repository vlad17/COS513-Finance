On the whole, predicting any financial market has proven to be difficult. Simon's work encapsulates many of the inherent problems. Price predictions might accurately predict directionality changes or general trends, but if these predictions consistently lag behind actual prices, they will not be useful despite a low RMSE. Model evaluation therefore must either demonstrate ability to generate profit through a trading scheme or demonstrate an ability to correctly predict directionality for desired trading period times and not merely a low RMSE. Simon identifies a training period of 3 years and a test period of 6 months as optimal for reducing predicted errors \cite{forex_neuralnets}.

\noindent In an examination of the potential for technical approaches to predict pricing movements using a Naive Bayes classifier, Gidofalvi found that a 20 minute period before and after the release of financial news allowed for a weak prediction of price movements \cite{gidofalvi2001using}. It was found by McQueen and Roley that fundamental macroeconomic news has little impact on stock prices but that other news types have effects dependent on responses of expected flows relative to equity discount rates \cite{mcqueen1993stock}.\\

\noindent Using textual analysis and SVMs for prediction, Schumaker and Chen were able to demonstrate 57\% directional accuracy when using breaking financial news articles to predict S\&P500 stock movements within 20 minute periods \cite{schumaker2009textual}. Bollen et al also achieved an accuracy of 86.7\% when incorporating semantic mood data from Twitter and using self-organizing fuzzy neural nets \cite{bollen2011twitter}.

\noindent A new area for price prediction is outlined by Phua et al, who validate the mostly non-financial GDELT as a valid news source for market prediction. This is in contrast to the financial data sets used by Schumaker and Chen, which is comprised of USecurity and Exchange Commission reports, stock related information such as from The Motley Fool, buy/sell/hold recommendations, and other similar news. GDELT does not have the numeric economic data we wish to avoid. Phua et al's topic analysis shows that GDELT consistently identifies impactful events on stock markets, despite its lack of financial focus. Term extraction from news sources is shown to be relevant using concept link exploration. However, Phua et al do find that not all significant events can be distinguished by GDELT. For example, the first riots in Singapore in 40 years did not appear significantly different than other news clusters. They were also unable to verify the quality of semantic score assignments. While they use decision trees to determine the factors potentially useful in price forecasting, they do not attempt to actually forecast prices in contrast to this paper.