On the whole, predicting any financial market has proven to be difficult. Simon's work encapsulates many of the inherent problems. Model evaluation cannot rely on using normal RMSE metrics, which might suffer from lag problems, but must either demonstate ability to generate profit through a trading scheme or demonstrate an ability to correctly predict directionality for desired trading period times. Simon identifies a training period of 3 years and a test period of 6 months as optimal.\cite{forex_neuralnets}

\noindent In an examination of the potential for technical approaches to predict pricing movements, Gidofalvi found that a 20 minute period before and after the release of financial news allowed for a weak prediction of price movements.\cite{gidofalvi2001using} It was found by McQueen and Roley that fundamental macroeconomic news has little impact on stock prices but that other news types have effects dependent on responses of expected flows relative to equity discount rates.\cite{mcqueen1993stock}\\

\noindent In addition to Phua et al's work in identifying crisis points in Singapore's stock market, Schumaker and Chen were able to demonstrate 57\% directional accuracy when using news articles to predict S\&P500 stocks in Gidofalvi's 20 minute period.\cite{schumaker2009technical} Bollen et al also achieved an accuracy of 86.7\% when incorporating semantic mood data from Twitter.\cite{bollen2011twitter} Other research has focused on extracting semantics from blogs or financial news. Hagenau et al show that context-specific feature extraction from news can reduce overfitting of models.\cite{hagenau2012automated}\\

\noindent The work done by Phua et al is important for validating GDELT in particular as a valid news source for market prediction. Topic analysis shows that GDELT includes impactful events across multiple iterations of clustering (thus correctly assigning impactful events large magnitudes) and also correctly chronologically tracks changing magnitudes. Term extraction from news sources is shown to be relevant using concept link exploration. Phua et all do find that not all significant events can be discovered by GDELT. They were also unable to verify that quality for semantic score assignments was high.