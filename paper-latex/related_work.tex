On the whole, predicting any financial market has proven to be difficult. Simon's work encapsulates many of the inherent problems. Model evaluation cannot rely on using normal RMSE metrics, which might suffer from lag problems, but must either demonstrate ability to generate profit through a trading scheme or demonstrate an ability to correctly predict directionality for desired trading period times. Simon identifies a training period of 3 years and a test period of 6 months as optimal for reducing predicted errors \cite{forex_neuralnets}.

\noindent In an examination of the potential for technical approaches to predict pricing movements, Gidofalvi found that a 20 minute period before and after the release of financial news allowed for a weak prediction of price movements \cite{gidofalvi2001using}. It was found by McQueen and Roley that fundamental macroeconomic news has little impact on stock prices but that other news types have effects dependent on responses of expected flows relative to equity discount rates \cite{mcqueen1993stock}.\\

\noindent In addition to Phua et al's work in identifying crisis points in Singapore's stock market, Schumaker and Chen were able to demonstrate 57\% directional accuracy when using news articles to predict S\&P500 stock movements within 20 minute periods \cite{schumaker2009textual}. Bollen et al also achieved an accuracy of 86.7\% when incorporating semantic mood data from Twitter \cite{bollen2011twitter}.

\noindent The work done by Phua et al is important for validating GDELT in particular as a valid news source for market prediction. Their topic analysis shows that GDELT consistently clusters impactful events together. Term extraction from news sources is shown to be relevant using concept link exploration. Phua et al do find that not all significant events can be distinguished by GDELT. For example, the first riots in Singapore in 40 years did not appear significantly different than other news clusters. They were also unable to verify the quality of semantic score assignments.